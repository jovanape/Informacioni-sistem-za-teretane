% !TEX encoding = UTF-8 Unicode
\documentclass[a4paper]{article}

\usepackage{color}
\usepackage{url}
\usepackage[T2A]{fontenc} % enable Cyrillic fonts
\usepackage[utf8]{inputenc} % make weird characters work
\usepackage{graphicx}
\usepackage[section]{placeins}

\usepackage[english,serbian]{babel}
%\usepackage[english,serbianc]{babel} %ukljuciti babel sa ovim opcijama, umesto gornjim, ukoliko se koristi cirilica

\usepackage[unicode]{hyperref}
\hypersetup{colorlinks,citecolor=green,filecolor=green,linkcolor=blue,urlcolor=blue}

\usepackage{listings}
\usepackage{subfiles}

\usepackage{tabularx}
\usepackage{longtable}

%\newtheorem{primer}{Пример}[section] %ćirilični primer
\newtheorem{primer}{Primer}[section]

\definecolor{mygreen}{rgb}{0,0.6,0}
\definecolor{mygray}{rgb}{0.5,0.5,0.5}
\definecolor{mymauve}{rgb}{0.58,0,0.82}

\lstset{ 
  backgroundcolor=\color{white},   % choose the background color; you must add \usepackage{color} or \usepackage{xcolor}; should come as last argument
  basicstyle=\scriptsize\ttfamily,        % the size of the fonts that are used for the code
  breakatwhitespace=false,         % sets if automatic breaks should only happen at whitespace
  breaklines=true,                 % sets automatic line breaking
  captionpos=b,                    % sets the caption-position to bottom
  commentstyle=\color{mygreen},    % comment style
  deletekeywords={...},            % if you want to delete keywords from the given language
  escapeinside={\%*}{*)},          % if you want to add LaTeX within your code
  extendedchars=true,              % lets you use non-ASCII characters; for 8-bits encodings only, does not work with UTF-8
  firstnumber=1000,                % start line enumeration with line 1000
  frame=single,	                   % adds a frame around the code
  keepspaces=true,                 % keeps spaces in text, useful for keeping indentation of code (possibly needs columns=flexible)
  keywordstyle=\color{blue},       % keyword style
  language=Python,                 % the language of the code
  morekeywords={*,...},            % if you want to add more keywords to the set
  numbers=left,                    % where to put the line-numbers; possible values are (none, left, right)
  numbersep=5pt,                   % how far the line-numbers are from the code
  numberstyle=\tiny\color{mygray}, % the style that is used for the line-numbers
  rulecolor=\color{black},         % if not set, the frame-color may be changed on line-breaks within not-black text (e.g. comments (green here))
  showspaces=false,                % show spaces everywhere adding particular underscores; it overrides 'showstringspaces'
  showstringspaces=false,          % underline spaces within strings only
  showtabs=false,                  % show tabs within strings adding particular underscores
  stepnumber=2,                    % the step between two line-numbers. If it's 1, each line will be numbered
  stringstyle=\color{mymauve},     % string literal style
  tabsize=2,	                   % sets default tabsize to 2 spaces
  title=\lstname                   % show the filename of files included with \lstinputlisting; also try caption instead of title
}
\begin{document}

\title{Informacioni sistem za teretane\\ \small{Seminarski rad u okviru kursa\\Informacioni sistemi\\ Matematički fakultet}}

\author{Selena Vukadinović, Tamara Ivanović, Jana Jovičić, Jovana Pejkić}

%\date{9.~april 2015.}

\maketitle
\newpage

\tableofcontents
\newpage

\section{Uvod}
\label{sec:uvod}
\subfile{sections/uvod}

%\newpage
\section{Dijagrami}
\subfile{sections/dijagrami}

\section{Slučajevi upotrebe}

\subsection{Slučajevi upotrebe korisnika}
\subsubsection{Slučajevi upotrebe neregistrovanog korisnika}
\paragraph{Registracija online}
\subfile{sections/registracija_neregistrovanog_korisnika}
\paragraph{Ograničen pristup informacijama}
\subfile{sections/ogranicen_pristup_informacijama_neregistrovanog_korisnika}

\subsubsection{Slučajevi upotrebe registrovanog korisnika}
\paragraph{Online plaćanje za registrovanog korisnika}
\subfile{sections/online_placanje_za_registrovanog_korisnika}

\subsection{Aktivnosti neregistrovanog korisnika}
\subsubsection{Registracija korisnika uživo}
\subfile{sections/uzivo_registracija}

\subsection{Aktivnosti klijenta}
\subsubsection{Plaćanje članarine uživo}
\subfile{sections/placanje_uzivo}

\subsubsection{Evidencija dolaska na trening}
\subfile{sections/evidencija_prisustva}

\newpage
\subsection{Aktivnosti vezane za personalne treninge}
\subfile{sections/personalni_treninzi}

\newpage
\subsection{Aktivnosti vezane za grupne treninge}

\newpage
\subsection{Aktivnosti administratora}
\subsubsection{Slučaj upotrebe: Registracija novog zaposlenog}
\subfile{sections/registracija_zaposlenog}

\subsubsection{Slučaj upotrebe: Dodavanje nove lokacije}
\subfile{sections/nova_lokacija}

\subsubsection{Slučaj upotrebe: Registracija vrste treninga}
\subfile{sections/nova_vrsta_treninga}

\newpage
%%%%%%%%%%%%%Takmičenje
\subsection{Takmičenje}
Takmičenje predstavlja poseban slučaj upotrebe koji nije uvek aktivan. Ukoliko administrator zakaže novo takmičenje u informacionom sistemu se omogućava pristup prijavi. 

\subsubsection{Kreiranje takmičenja}
\subfile{sections/takmicenje/kreiranje_takmicenja}
\subsubsection{Prijava za sudiju}
\subfile{sections/takmicenje/prijava_sudija}
\subsubsection{Prijava za takmičenje uživo}
\subfile{sections/takmicenje/prijava_takmicenje_uzivo}

%\subsubsection{Prijava za takmičenje online}
%\subsubsection{Otkazivanje učešća uživo}
%\subsubsection{Otkazivanje učešća online}
\subsubsection{Izmena podataka o takmičenju}
\subfile{sections/takmicenje/izmena_takmicenja}
\subsubsection{Otkazivanje takmičenja}
\subfile{sections/takmicenje/otkazivanje_takmicenja}

\end{document}
