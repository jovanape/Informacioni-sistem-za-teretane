\documentclass[../main.tex]{subfiles}

\begin{document}

\subsection{Analiza sitema}

Tokom godina broj teretana se rapidno povećava i neretko se osnivači odlučuju na korak proširenja svojih kapaciteta i samim tim posedovanja lanca teretana. Ideja informacionog sistema za teretane je da omogući efikasnije organizovanje i praćenje aktivnosti vlasniku, kao i udobnije organizovanje svojih sportskih aktivnosti za osobe koje žele da vežbaju. Ovaj sistem omogućava klijentima da se registruju i izvrše uplatu članarine bilo uživo ili online preko sistema, zakažu trening po izboru i vide pregled svojih aktivnosti. Ukoliko korisnik nije siguran da li želi da se pridruži ovom lancu teretana ima mogućnost da kao gost vidi osnovne informacije vezane za treninge, trenere i planove koji postoje. Administrator ima mogućnosti da kreira i briše naloge svojih zaposlenih, kao i da šalje obaveštenja ostalim korisnicima čime je omogućeno udobnije korišćenje jer korisnici ne moraju da proveravaju non stop sistem. Treneri imaju udobniji način za praćenje i izmenu svojh treninga čime se olakšava njihova organizovanost. Zaposleni na recepciji svoja zaduženja izvršava mnogo lakše zahvaljujući mogućnosti da sve uplate, nove naloge, zakazivanje termina i praćenje dolazaka beleži na jednom mestu, upravo u ovom informacionom sistemu.

\subsection{Učesnici sistema}
Ovaj informacioni sistem karakteriše pet vrsta korisnika, od kojih su četiri registrovana i jedan neregistrovan. U registrovane korisnike spadaju sve osobe koje su ili zaposlene u lancu teretana ili dolaze da vežbaju tu. Neregistrovani korisnik je svaka osoba koja želi da se informiše o treninzima koji se održavaju, trenerima koji su zaposleni tu i opštim informacijama o teretani.
\begin{enumerate}
  \item \textbf{Trener} – zakazuje i otkazuje termine, uređuje svoj profil i može da izmeni informacije o treninzima koji su zakazani
  \item \textbf{Klijent} – registruje se, plaća članarinu i zakazuje termine ( sve ovo online ili preko recepcije)
  \item  \textbf{Zaposleni na recepciji} – registruje klijente, zakazuje termine, prima uplate i vodi evidenciju o dolascima
  \item \textbf{Administrator} – kreira naloge za nove zaposlene, briše naloge starih zaposlenih, dodaje nove lokacije teretane nakon otvaranja i ima mogućnost slanja važnih obaveštenja
  \item \textbf{Gost} – dobija najosnovniji prikaz termina i informacija o teretanama
\end{enumerate}



\end{document}