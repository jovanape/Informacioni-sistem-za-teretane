\documentclass[../main.tex]{subfiles}

\begin{document}

\subsection{Analiza sitema}

Tokom godina broj teretana se rapidno povećava i neretko se osnivači odlučuju na korak proširenja svojih kapaciteta i samim tim posedovanja lanca teretana. Ideja informacionog sistema za teretane je da omogući efikasnije organizovanje i praćenje aktivnosti vlasniku, kao i udobnije organizovanje svojih sportskih aktivnosti za osobe koje žele da vežbaju. Ovaj sistem omogućava klijentima da se registruju i izvrše uplatu članarine bilo uživo ili online preko sistema, zakažu trening po izboru i vide pregled svojih aktivnosti. Ukoliko korisnik nije siguran da li želi da se pridruži ovom lancu teretana ima mogućnost da kao gost vidi osnovne informacije vezane za treninge, trenere i planove koji postoje. Administrator ima mogućnosti da kreira i briše naloge svojih zaposlenih, kao i da šalje obaveštenja ostalim korisnicima čime je omogućeno udobnije korišćenje jer korisnici ne moraju da proveravaju non stop sistem. Treneri imaju udobniji način za praćenje i izmenu svojh treninga čime se olakšava njihova organizovanost. Zaposleni na recepciji svoja zaduženja izvršava mnogo lakše zahvaljujući mogućnosti da sve uplate, nove naloge, zakazivanje termina i praćenje dolazaka beleži na jednom mestu, upravo u ovom informacionom sistemu.

Ovaj sistem je kreiran za lanac teretana koji je organizovan tako da svaka od teretana iz tog lanca ima prostor opremljen fitnes spravama kao i sale za grupne treninge, čime omogućuje svojim klijentima izbor načina treniranja. Klijent može odabrati individualne treninge tokom kojih će ga stalno nadgledati jedan od personalnih trenera, ili se može odlučiti da vežba samostalno. A ukoliko mu ne odgovara nijedan od ova dva načina vežbanja, može izabrati da prisustvuje grupnim treninzima poput aerobika, TBW-a, pilatesa i sličnim treninzima koji se organizuju u salama za grupne treninge. 

Pored osnovnih stvari vezanih za treniranje i uplaćivanje članarina, ove teretane organizuju takmičenja i programe obuke za dobijenje licence za trenera. Takođe, roditeljima omogućuju da, dok oni prisustvuju treningu, svoju decu ostave u igraonicama koje se nalaze na svakoj lokaciji.
% Mozda opisati detaljnije ove tri oblasti? Ili bolje da ostane samo ovako kratko, posto su slucajevi upotrebe svakako detaljni?

\subsection{Učesnici sistema}
Ovaj informacioni sistem karakteriše šest vrsta korisnika, od kojih je pet registrovano, a jedan je neregistrovan. U registrovane korisnike spadaju sve osobe koje su ili zaposlene u lancu teretana ili dolaze da vežbaju tu. Neregistrovani korisnik je svaka osoba koja želi da se informiše o treninzima koji se održavaju, trenerima koji su zaposleni tu i opštim informacijama o teretani.
\begin{enumerate}
  %\item \textbf{Trener} – zakazuje i otkazuje termine, uređuje svoj profil i može da izmeni informacije o treninzima koji su zakazani
  \item \textbf{Personalni trener} \\
  Personalni trener je osoba koja je zadužena da tokom treninga radi samo sa jednim klijentom. Svaki personalni trener ima sopstveni raspored treninga u kom svake nedelje navodi termine u kojima može da primi klijente. Takođe, može i da ukloni prethodno dodati slobodan termin. Personalni trener može da testira napredak klijenta i na osnovu toga da mu formira novi plan vežbanja i ishrane.
  \item \textbf{Trener grupnih treninga} \\
  %\item \textbf{Klijent} – registruje se, plaća članarinu i zakazuje termine ( sve ovo online ili preko recepcije)
  \item \textbf{Klijent} \\
  Klijent ima mogućnost da bira vrstu članarine kao i tip treninga na koji želi da ide. Omogućeno mu je plaćanje članarine preko sistema, ali može odabrati da je plati i uživo na recepciji ukoliko mu je tako lakše.\\ 
  Ukoliko se odluči za personalne treninge, prilikom prijave za trening može odabrati trenera kao i neki od termina iz njegovog rasporeda. Prijavu može izvršiti online ili na recepciji.\\
  Ako je odabrao da ide na grupne treninge, i u tom slučaju potrebno je da prijavi kom treningu će prisustvovati. To može da uradi na recepciji ili online direktno sa stranice rasporeda za teretanu u kojoj trenira. 
  
  \item  \textbf{Recepcionar} \\ Zaposleni na recepciji koji vrši registraciju korisnika, zakazuje termine, prima uplate i vodi evidenciju o dolascima.
  \item \textbf{Administrator}\\
  Administrator je neko iz uprave lanca teretana. On može da kreira naloge za nove zaposlene, briše naloge starih zaposlenih, dodaje nove lokacije teretane nakon otvaranja i ima mogućnost slanja važnih obaveštenja. Aministrator takođe može da osmisli i novu vrstu grupnog treninga, da ga doda u raspored neke od teretana iz lanca, kao i da odredi trenera koji će voditi taj trening.
  \item \textbf{Neregistrovani korisnik} \\
  Neregistrovani korisnik dobija najosnovniji prikaz informacija o teretanama, trenerima koji su zaposleni u njima i terminima treninga. Ukoliko odluče da se učlane, to mogu učiniti online ili na recepciji nekog od centara.  
\end{enumerate}






\end{document}