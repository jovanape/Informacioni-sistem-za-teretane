\documentclass[../main.tex]{subfiles}
\begin{document}

%\begin{center}
%\begin{tabularx}{\textwidth}{|1|X|}
\begin{longtable}{| p{.20\textwidth} | p{.80\textwidth} |} 

\hline
    Kratak opis &  Administrator unosi novu vrstu treninga na sistem.\\ 
\hline    
    Učesnici & 
    	\begin{enumerate}
        \item Administrator - želi da doda novu vrstu treninga na sistem.
     \end{enumerate}\\
\hline
   Preduslovi & \begin{enumerate}
       \item Sistem je u funkciji.
       \item Administrator ima sve potrebne informacije o novoj vrsti treninga.
   \end{enumerate}\\
\hline  
    Postuslovi & \begin{enumerate}
        \item Nova vrsta treninga je uspešno dodata u sistem.
        \item Treneri su u mogućnosti da zakazuju treninge ove vrste.
    \end{enumerate}\\
\hline
    Osnovni tok & \begin{enumerate}
        \item Administrator bira opciju za dodavanje nove vrste treninga.
        \item Administrator popunjava formu za unos podataka o vsti novog treninga.
        \item Nakon što su popunjena sva polja, bira opciju „Dodaj vrstu treninga“.
        \item Sistem unosi unete podatke o novoj vrsti treninga u bazu.
        \item Sistem obaveštava administratora da li je nova vrsta treninga uspešno dodata.
    \end{enumerate}\\
\hline
    Alternativni tokovi & \begin{itemize}
        \item[A4]  Sistem nije u mogućnosti da doda novu vrstu treninga. Tada on obaveštava administratora o neuspehu. Slučaj upotrebe se završava.

    \end{itemize}\\
\hline
    Podtokovi & /\\
\hline
    Specijalni zahtevi & /\\
\hline
    Dodatne informacije & Potrebni podaci za novu vrstu treninga su naziv, tip(fitness, body building, joga,..), indikacija da li je grupni ili personalizovani .\\
\hline
%\end{tabularx}
\caption{Dodavanje nove vrste treninga} % needs to go inside longtable environment
%\label{tab:myfirstlongtable}
\end{longtable}
%\end{center}  
\end{document} 