\documentclass[../main.tex]{subfiles}
\begin{document}

%\begin{center}
%\begin{tabularx}{\textwidth}{|1|X|}
\begin{longtable}{| p{.20\textwidth} | p{.80\textwidth} |} 

\hline
    Kratak opis &  Administrator menja parametre postojeće lokacije teretane u sistemu.\\ 
\hline    
    Učesnici & 
    	\begin{enumerate}
        \item Administrator - želi da ažurira lokaciju.
     \end{enumerate}\\
\hline
   Preduslovi & \begin{enumerate}
       \item Sistem je u funkciji.
       \item Administrator ima sve potrebne informacije o lokaciji, tj. adresu i grad teretane.
   \end{enumerate}\\
\hline  
    Postuslovi & \begin{enumerate}
        \item Baza je ažuirana i sadrži nove vrednosti parametara lokacije teretane.
    \end{enumerate}\\
\hline
    Osnovni tok & \begin{enumerate}
        \item Administrator pristupa stranici za pretragu lokacija teretane.
        \item Administrator popunjava formular pretrage unoseći adresu i grad u kojem se teretana nalazi.
        \item Nakon što su polja popunjena, administrator bira opciju „Pretraži lokacije“.
        \item Administrator pronalazi željenu lokaciju i bira opciju „Izmeni lokaciju“.
        \item Sistem ažurira vrednosti promenjenih parametara zadate lokacije u bazi.
        \item Sistem obaveštava administratora da je lokacija uspešno izmenjena.
    \end{enumerate}\\
\hline
    Alternativni tokovi & \begin{itemize}
        \item[A6]  Sistem ne uspeva da ažurira lokaciju. Tada on obaveštava administratora o neuspehu i javlja mu da pokuša ponovo istu operaciju. Slučaj upotrebe se nastavlja na koraku 4..

    \end{itemize}\\
\hline
    Podtokovi & /\\
\hline
    Specijalni zahtevi & /\\
\hline
    Dodatne informacije & /\\
\hline
%\end{tabularx}
\caption{Ažuriranje postojeće lokacije teretane} % needs to go inside longtable environment
%\label{tab:myfirstlongtable}
\end{longtable}
%\end{center}  
\end{document} 