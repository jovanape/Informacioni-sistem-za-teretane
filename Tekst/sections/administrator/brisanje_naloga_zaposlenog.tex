\documentclass[../main.tex]{subfiles}
\begin{document}

%\begin{center}
%\begin{tabularx}{\textwidth}{|1|X|}
\begin{longtable}{| p{.20\textwidth} | p{.80\textwidth} |} 

\hline
    Kratak opis &  Administrator briše nalog zaposlenog iz sistema pristupajući nalogu i brišući sve informacije koje nalog sadrži.\\ 
\hline    
    Učesnici & 
    	\begin{enumerate}
        \item Administrator - želi da ukloni nalog zaposlenog.
     \end{enumerate}\\
\hline
   Preduslovi & \begin{enumerate}
       \item Sistem je u funkciji.
       \item Administrator ima sve potrebne informacije o zaposlenom, tj. korisničko ime i JMBG zaposlenog.
   \end{enumerate}\\
\hline  
    Postuslovi & \begin{enumerate}
        \item Baza je ažuirana, tj. nalog zaposlenog je uklonjen iz sistema.
    \end{enumerate}\\
\hline
    Osnovni tok & \begin{enumerate}
        \item Administrator pristupa stranici za pretragu naloga zaposlenih.
        \item Administrator popunjava formular pretrage unoseći korisničko ime zaposlenog.
        \item Nakon što su polja popunjena, administrator bira opciju „Pretraži zaposlene“.
        \item Administrator pronalazi željeni nalog i bira opciju „Ukloni zaposlenog“.
        \item Sistem briše nalog, kao i sve informacije nalog sadrži.
        \item Sistem obaveštava administratora da je nalog uspešno uklonjen.
    \end{enumerate}\\
\hline
    Alternativni tokovi & \begin{itemize}
        \item[A4]  Pretragom se ne pronalazi ni jedan ili se pronalazi više od jednog željenog naloga, nakon čega administrator unosi JMBG zaposlenog i pretragu vrši po drugom parametru, JMBG-u, umesto po korisničkom imenu. Slučaj upotrebe se nastavlja na koraku 3.
        \item[A6]  Sistem ne uspeva da obriše nalog zaposlenog. Tada on obaveštava administratora o neuspehu i javlja mu da pokuša ponovo istu operaciju. Slučaj upotrebe se nastavlja na koraku 4..

    \end{itemize}\\
\hline
    Podtokovi & /\\
\hline
    Specijalni zahtevi & /\\
\hline
    Dodatne informacije & /\\
\hline
%\end{tabularx}
\caption{Uklanjanje naloga zaposlenog u teretani} % needs to go inside longtable environment
%\label{tab:myfirstlongtable}
\end{longtable}
%\end{center}  
\end{document} 