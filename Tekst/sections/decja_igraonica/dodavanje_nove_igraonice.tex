\documentclass[../main.tex]{subfiles}
\begin{document}

%\begin{center}
%\begin{tabularx}{\textwidth}{|1|X|}
\begin{longtable}{| p{.20\textwidth} | p{.80\textwidth} |} 

\hline
    Kratak opis & Administrator unosi novu igraonicu u sistem. \\ 
\hline    
    Učesnici & 
    	\begin{itemize}
        \item Administrator koji želi da doda novu igraonicu u sistem.
     \end{itemize}\\
\hline
   Preduslovi & \begin{itemize}
       \item Sistem je u funkciji.
       \item Administrator je prijavljen i autorizovan za korišćenje sistema.
       \item Administrator ima sve neophodne informacije o novoj igraonici.
   \end{itemize}\\
\hline  
    Postuslovi & \begin{itemize}
        \item Nova igraonica je uspešno dodata u sistem.
        \item Baza podataka je ažurirana.
        \item Klijenti mogu da kupe termine za novu igraonicu, a potom i da koriste njene usluge.
    \end{itemize}\\
\hline
    Osnovni tok & \begin{enumerate}
        \item Administrator bira opciju za dodavanje nove igraonice.
        \item Sistem prikazuje formular sa pitanjima kao što su lokacija igraonice, tip igraonice i drugo.
        \item Administrator popunjava formular neophodnim podacima.
        \item Administrator potvrđuje unos podataka i dodavanje nove igraonice. % bice neko dugme
        \item Sistem proverava unete podatke - na primer da li na unetoj lokaciji već postoji igraonica, da li na toj lokaciji postoji teretana, itd. % igraonica moze da se doda samo tamo gde vec postoji teretana i ne postoji vec jedna takva (vrsta) igraonica; moze pri popunjavanju forme da se napravi padajuca lista odakle se bira adresa od vec postojecih, tj. lokacija, kako bi se sprecio unos nepostojece lokacije ili pogresne
        \item Sistem obaveštava administratora da je nova igraonica uspešno dodata.
        \item Ažurira se baza podataka.
        % TODO: Da li treba da piše Sistem ažurira bazu podataka ili je ok ovako?
    \end{enumerate}\\
\hline
    Alternativni tokovi & \begin{itemize}
        \item[A3-A4] Administrator je odustao od dodavanja nove igraonice: Slučaj upotrebe se završava.
        % Izmene nisu sačuvane, ali pošto nisu mislim da nema potrebe naglašavati to.
        \item[A6] Podaci koje je administrator uneo u formular nisu validni (ne zadovoljavaju određene uslove): Slučaj upotrebe se nastavlja u koraku 3.
        % TODO: Da li je ovo slucaj A6 ili treba A5 - slicna situacija je bila i u drugim tabelama slucajeva upotrebe?
        \item[A6] Sistem nije u mogućnosti da doda novu igraonicu: Sistem obaveštava administratora o neuspehu. Slučaj upotrebe se završava.
    \end{itemize}\\
\hline
    Podtokovi & /\\
\hline
    Specijalni zahtevi & /\\
\hline
    Dodatne informacije & Potrebni podaci za novu igraonicu su tip igraonice, adresa na kojoj već postoji teretana ali ne i taj konkretan tip igraonice.
    % mozda jos neki prateci podaci
    \\
\hline
%\end{tabularx}
\caption{Dodavanje nove igraonice} % needs to go inside longtable environment
%\label{tab:myfirstlongtable}
\end{longtable}
%\end{center}  
\end{document} 