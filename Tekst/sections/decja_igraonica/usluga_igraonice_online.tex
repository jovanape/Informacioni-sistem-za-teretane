\documentclass[../../main.tex]{subfiles}

\begin{document}


\begin{longtable}{| p{.20\textwidth} | p{.80\textwidth} |} 
\hline
    Kratak opis & Klijent želi da koristi usluge igraonice - online. Klijent bira vrstu igraonice, broj termina, kao i druge opcije, ukoliko ima dete/dece i želi da ono/ona boravi/borave u igraonici za vreme njegovog treninga.  \\ 
\hline    
    Učesnici &
    \begin{enumerate}
        \item Klijent
    \end{enumerate}\\
\hline
   Preduslovi &
   \begin{enumerate}
        \item Sistem je u funkciji
        \item Klijent je registrovan u sistemu
        \item Klijent je prijavljen na sistem
    \end{enumerate}\\
\hline  
    Postuslovi & 
    \begin{enumerate}
        \item Klijent je kupio termine za određenu vrstu igraonice u nekoj od teretana i može da krene sa korišćenjem ove usluge.
    \end{enumerate} \\
\hline
    Osnovni tok & 
    \begin{enumerate}
        \item Klijent odlazi na stranicu posvećenu igraonicama za decu.
        \item Klijent, u skladu sa lokacijom igraonice odnosno teretane, bira vrstu igraonice čije usluge želi da koristi.
        \item Klijent čita detaljne informacije o izabranoj igraonici kao i o tome koji uslovi moraju biti ispoštovani da bi se usluga koristila.
        \item Klijent bira opciju 'Kupi termine'.
        \item Sistem prikazuje formular koji klijent treba da popuni.
        \item Klijent popunjava formular tako što unosi broj dece, broj termina i ostale tražene podatke.
        \item Sistem proverava da li su uneti podaci ispravni, računa ukupnu cenu na osnovu unetih podataka i ispisuje je klijentu.
        \item Klijent izvršava plaćanje klikom na dugme za potvrdu plaćanja.
        \item Sistem proverava unete podatke i raspoloživa sredstva korisnika.
        \item Sistem šalje potvrdu da je plaćanje uspešno izvršeno.
        \item Ažurira se baza podataka.
    \end{enumerate}\\
\hline
    Alternativni tokovi & 
    \begin{itemize}
        \item[A6, A8] Klijent bira opciju za otkazivanje započetog procesa: Slučaj upotrebe se nastavlja u koraku 1.
        \item[A7] Podaci koje je klijent uneo nisu ispravni ili nisu potpuni: Sistem nije uspeo da nastavi računanje ukupne cene. Slučaj se nastavlja u koraku 6.
        \item[A9] Podaci koje je klijent uneo nisu ispravni ili nisu potpuni: Sistem nije uspeo da nastavi proces plaćanja. Slučaj se nastavlja u koraku 6.
        \item[A9] Klijent nema dovoljno sredstava na računu: Sistem obaveštava klijenta da nema dovoljno sredstava na računu. Klijentu se pruža prilika da unese broj (i ostale parametre) druge platne kartice (ukoliko takvu poseduje) i da sa njom pokuša plaćanje ili da smanji broj termina, odnosno odabere neku jeftiniju opciju. Slučaj se nastavlja na koraku 6.
        % u koraku 6 (8) klijent može i da odustane
        \item[A9] Problem pri povezivanju sistema i banke: Klijent se obaveštava o situaciji i poziva da pokuša ponovo plaćanje. Slučaj se nastavlja u koraku 3.
    \end{itemize} \\
\hline
    Podtokovi & /\\
\hline
    Specijalni zahtevi & /\\
\hline
    Dodatne informacije &
    \begin{enumerate}
        \item Informacije koje klijent unosi u formular su ime, prezime, broj članske karte, tip igraonice, broj termina, broj dece, broj platne kartice korisnika, datum isteka platne kartice i CVV kod platne kartice.
        % možda kontakt roditelja, mada već postoji u sistemu?
        % bira tip igraonice: igraonica za malu decu ili za igranje video igrica (za sada)
    \end{enumerate}\\
\hline
\caption{Usluga igraonice online}
\end{longtable}

\end{document}