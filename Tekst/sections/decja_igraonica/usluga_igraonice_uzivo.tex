\documentclass[../../main.tex]{subfiles}

\begin{document}


\begin{longtable}{| p{.20\textwidth} | p{.80\textwidth} |} 
\hline
    Kratak opis & Klijent želi da koristi usluge igraonice - online. Klijent bira vrstu igraonice, broj termina, kao i druge opcije, ukoliko ima dete/dece i želi da ono/ona boravi/borave u igraonici za vreme njegovog treninga.  \\ 
\hline
    Učesnici &
    \begin{enumerate}
        \item Klijent
        \item Recepcioner
    \end{enumerate}\\
\hline
   Preduslovi &
    \begin{enumerate}
        \item Sistem je u funkciji
        \item Klijent je registrovan u sistemu
        \item Klijent je prijavljen na sistem
        \item Recepcioner je prijavljen na sistem
    \end{enumerate}\\
\hline  
    Postuslovi & 
    \begin{enumerate}
         \item Klijent je kupio termine za određenu vrstu igraonice u nekoj od teretana i može da krene sa korišćenjem ove usluge.
    \end{enumerate} \\
\hline
    Osnovni tok & 
    \begin{enumerate}
        \item Klijent odlazi na recepciju.
        \item Klijent govori recepcioneru usluge koje vrste igraonice (u skladu sa lokacijom igraonice odnosno teretane) želi da koristi.
        \item Recepcioner obaveštava klijenta o opštim informacijama vezanim za izabranu igraonicu kao i o tome koji uslovi moraju biti ispoštovani da bi se usluga odabrane igraonice koristila.
        \item Klijent je saglasan sa tim da kupi termine za odabranu igraonicu
        \item Recepcioner bira opciju 'Kupi termine' za odabranu igraonice
        \item Sistem prikazuje formular koji recepcioner treba da popuni.
        \item Recepcioner klijentu postavlja pitanja iz formulara i beleži odgovore (broj dece, broj termina itd.)
        \item Sistem proverava da li su uneti podaci ispravni, računa ukupnu cenu na osnovu unetih podataka i ispisuje je (recepcioneru). Recepcioner saopštava cenu klijentu.
        \item Klijent izvršava plaćanje (gotovinom ili platnom karticom). %9.
        \item Recepcioner unosi podatak (u sistem) da su termini uspešno uplaćeni. %10.
        \item Sistem proverava (do sada) unete podatke.
        \item Sistem šalje potvrdu da je plaćanje uspešno izvršeno.
        \item Ažurira se baza podataka.
        \item Sistem štampa račun.
        \item Recepcioner obaveštava klijenta da su termini za igraonicu uspešno uplaćeni i daje mu račun.
        \item Klijent odlazi sa računom.
    \end{enumerate}\\
\hline
    Alternativni tokovi & 
    \begin{itemize}
        \item[A4, A9] Klijent u potpunosti odustaje od započetog procesa kupovine termina za igraonicu: Recepcioner ostavlja sistem u stanju u kome je bio pre nego što je započet proces kupovine termina za igraonicu. Slučaj upotrebe se završava.
        % Da li da podelim na 2 alernativna toka jer samo ako je A9 recepcioner treba da ostavi sistem u stanju u kakvom je bio?
        \item[A8] Podaci koje je recepcioner uneo nisu ispravni ili nisu potpuni: Sistem nije uspeo da nastavi računanje ukupne cene. Slučaj se nastavlja u koraku 7.
        \item[A11] Podaci koje je recepcioner uneo nisu ispravni ili nisu potpuni: Sistem nije uspeo da nastavi proces plaćanja. Ako je recepcioner naplatio više ili manje nego što treba, reguliše se razlika u plaćanju između klijenta i recepcionera. Slučaj se nastavlja u koraku 10.
        % Da li je ok ovako ili treba da se vrati na korak 9?
        \item[A9] Klijent nema dovoljno sredstava na računu pri plaćanju karticom: Sistem obaveštava klijenta da nema dovoljno sredstava na računu. Klijentu se pruža prilika da plati drugom platnom karticom (ukoliko takvu poseduje) i da sa njom pokuša plaćanje ili da plati u gotovini ili da smanji broj termina, odnosno odabere neku jeftiniju opciju. Slučaj se nastavlja u koraku 9.
        % u ovom koraku (9) klijent može i da odustane
        \item[A9] Klijent nema dovoljno sredstava  pri plaćanju gotovinom: Recepcioner pruža priliku klijentu da plati platnom karticom (ukoliko takvu poseduje) ili da smanji broj termina, odnosno odabere neku jeftiniju opciju. Slučaj se nastavlja u koraku 9.
        % u ovom koraku (9) klijent može i da odustane
        \item[A12] Problem pri povezivanju sistema i banke kod plaćanja karticom: Recepcioner obaveštava klijenta o situaciji i poziva da pokuša ponovo plaćanje ili da plati u gotovini. Slučaj se nastavlja u koraku 9.
        % Da li ovde treba da bude korak A12?
    \end{itemize} \\
\hline
    Podtokovi & /\\
\hline
    Specijalni zahtevi & /\\
\hline
    Dodatne informacije &
    \begin{enumerate}
     \item Informacije koje klijent daje recepcioneru su ime, prezime, broj članske karte, tip igraonice, broj termina, broj dece.
    \end{enumerate}\\
\hline
\caption{Usluga igraonice uživo}
\end{longtable}

\end{document}