\documentclass[../../main.tex]{subfiles}

\begin{document}


\begin{longtable}{| p{.20\textwidth} | p{.80\textwidth} |} 
\hline
    Kratak opis & Klijent želi uživo da odabere igraonicu čije će usluge moći da koristi. Klijent bira vrstu igraonice, broj termina, kao i druge opcije, ukoliko ima dete/dece i želi da ono/ona boravi/borave u igraonici za vreme njegovog treninga.  \\ 
\hline
    Učesnici &
    \begin{enumerate}
        \item Klijent koji želi da odabere igraonicu.
        \item Recepcioner koji omogućava klijentu da odabere igraonicu.
    \end{enumerate}\\
\hline
   Preduslovi &
    \begin{enumerate}
        \item Sistem je u funkciji.
        \item Korisnik je registrovan u sistemu (korisnik je klijent).
        %\ item Klijent je prijavljen na sistem?
        \item Recepcioner je prijavljen na sistem i autorizovan za korišćenje sistema.
        %\ item Ukoliko klijent plaća karticom POS sistem je u fukciji.
        % Veza sa bankom je u funkciji (ukoliko se plaća karticom).
    \end{enumerate}\\
\hline  
    Postuslovi & 
    \begin{enumerate}
         \item Klijent je odabrao igraonicu koja mu odgovara i čije usluge želi da koristi.
         % \item Baza podataka je ažurirana.
         % \item Klijent dobija potvrdu o izvršenoj uplati.
    \end{enumerate} \\
\hline
    Osnovni tok & 
    \begin{enumerate}
        \item Klijent zahteva od recepcionera da mu omogući izbor igraonice.
        \item Recepcioner informiše klijenta o igraonicama koje su u ponudi.
        % o tipu igraonice i lokaciji
        \item Klijent govori recepcioneru usluge koje igraonice želi da koristi.
        \item Recepcioner obaveštava klijenta o detaljnim informacijama vezanim za izabranu igraonicu kao i o tome koji uslovi moraju biti ispoštovani da bi se usluga te igraonice koristila.
        \item Klijent je saglasan sa uslovima.
        \item Recepcioner bira opciju za kupovinu termina za odabranu igraonicu.
        \item Sistem prikazuje formular koji recepcioner treba da popuni.
        \item Recepcioner postavlja pitanja iz formulara klijentu i beleži odgovore (broj dece, broj termina, itd.).
        \item Recepcioner potvrđuje unos.
        \item Sistem proverava da li su uneti podaci ispravni, računa ukupnu cenu na osnovu unetih podataka i ispisuje je (recepcioneru).
        \item Recepcioner saopštava cenu klijentu.
        \item Klijent je uspešno odabrao igraonicu i saglasan je sa cenom. % nastavlja proces plaćanja
        
        % \item Klijent izvršava plaćanje (gotovinom ili platnom karticom). %9.
        % \item Recepcioner unosi podatak (u sistem) da su termini uspešno uplaćeni. %10.
        % \item Sistem proverava (do sada) unete podatke.
        % \item Sistem šalje potvrdu da je plaćanje uspešno izvršeno.
        % \item Ažurira se baza podataka.
        % \item Sistem štampa dokaz o uplati.
        % \item Recepcioner obaveštava klijenta da su termini za igraonicu uspešno uplaćeni i daje mu dokaz o uplati.
        % \item Klijent odlazi sa dokazom o uplati.
    \end{enumerate}\\
\hline
    Alternativni tokovi & 
    \begin{itemize}
        \item[A1-A12] Klijent u potpunosti odustaje od započetog procesa odabira igraonice: Recepcioner ostavlja sistem u stanju u kome je bio pre nego što je započet proces odabira igraonice. Slučaj upotrebe se završava.
        % Da li da podelim na 2 alernativna toka jer samo ako je A9 recepcioner treba da ostavi sistem u stanju u kakvom je bio?
        \item[A10] Podaci koje je recepcioner uneo nisu ispravni ili nisu potpuni: Sistem nije uspeo da nastavi računanje ukupne cene. Slučaj se nastavlja u koraku 7.
        % \item[A11] Podaci koje je recepcioner uneo nisu ispravni ili nisu potpuni: Sistem nije uspeo da nastavi proces plaćanja. Ako je recepcioner naplatio više ili manje nego što treba, reguliše se razlika u plaćanju između klijenta i recepcionera. Slučaj se nastavlja u koraku 10.
        % Da li je ok ovako ili treba da se vrati na korak 9?
        % \item[A9.2] Klijent nema dovoljno sredstava na računu pri plaćanju karticom: Recepcioner obaveštava klijenta da nema dovoljno sredstava na računu. Klijentu se pruža prilika da plati drugom platnom karticom (ukoliko takvu poseduje) i da sa njom pokuša plaćanje ili da plati u gotovini ili da smanji broj termina, odnosno odabere neku jeftiniju opciju. Slučaj se nastavlja u koraku 9.
        % u ovom koraku (9) klijent može i da odustane - već navedeno
        % \item[A9.1] Klijent nema dovoljno sredstava  pri plaćanju gotovinom: Recepcioner pruža priliku klijentu da plati platnom karticom (ukoliko takvu poseduje) ili da smanji broj termina, odnosno odabere neku jeftiniju opciju. Slučaj se nastavlja u koraku 9.
        % u ovom koraku (9) klijent može i da odustane - već navedeno
        % \item[A12] Problem pri povezivanju sistema i banke kod plaćanja karticom: Recepcioner obaveštava klijenta o situaciji i poziva da pokuša ponovo plaćanje ili da plati u gotovini. Slučaj se nastavlja u koraku 9.
        % Da li ovde treba da bude korak A12?
        %\item [A] Pad sistema u bilo kom trenutku tokom procesa kupovine termina za igraonicu/e: Potrebno je obezbediti da se sve transakcije mogu oporaviti.
        %\begin{itemize}
         %   \item Recepcionar resetuje sistem i bira oporavak sistema. Loguje se.
         %   \item Sistem rekonstruiše stanje koje je bilo.
         %   \item Recepcionar započinje proces (odabira i kupovine termina) iz početka.
        % \end{itemize}
        \item [A1 - A12] Klijent se predomislio oko izbora igraonice: Klijent obaveštava recepcionera da želi da promeni izbor. Recepcioner ostavlja sistem u stanju u kome je bio pre nego što je započet proces odabira igraonice. Slučaj se nastavlja u koraku 3.
    \end{itemize} \\
\hline
    Podtokovi & %\begin{itemize}
 %\item [9.1] Plaćanje gotovinom
 %\begin{enumerate}
 %\item Klijent predaje recepcionaru novac.
 %\item Recepcionar prebrojava novac i unosi sumu u sistem.
 %\item Sistem ispisuje koliki je kusur.
 %\item Recepcionar vraća kusur, ukoliko ga ima.
%\end{enumerate}
 
%\item [9.2] Plaćanje platnom karticom
%\begin{enumerate}
%\item Recepcioner bira opciju za plaćanje karticom.
%\item Klijent daje karticu recepcioneru.
%\item Recepcioner prislanja ili ubacuje karticu u aparat.
%\item Sistem upravlja plaćanjem i traži unos pin koda.
%\item Klijent unosi pin kod.
%\item Sistem proverava informacije.
%\item Sistem beleži informacije o uplati.
%\item Recepcioner vraća karticu klijentu.
%\end{enumerate}
% \end{itemize}
\\
\hline
    Specijalni zahtevi & Teretana poseduje POS aparat. \\
\hline
    Dodatne informacije &
    \begin{enumerate}
     \item Informacije koje klijent daje recepcioneru su ime, prezime, username, tip igraonice, broj termina, broj dece.
     % Da li treba username? Mislim da da jer sada klijent nije logovan, jer je uživo.
     %broj članske karte, 
    \end{enumerate}\\
\hline
\caption{Odabir igraonice uživo}
\end{longtable}

\end{document}