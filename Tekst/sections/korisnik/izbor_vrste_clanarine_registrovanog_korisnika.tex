\documentclass[../main.tex]{subfiles}
\begin{document}

%\begin{center}
%\begin{tabularx}{\textwidth}{|1|X|}
\begin{longtable}{| p{.20\textwidth} | p{.80\textwidth} |} 

\hline
    Kratak opis & Klijent bira vrstu članarine odnosno tip paketa koji će moći da koristi u ograničenom periodu. \\ 
\hline    
    Učesnici & 
    	\begin{enumerate}
        \item Klijent - želi da odabere paket koji mu najviše odgovara.
     \end{enumerate}\\
\hline
   Preduslovi & \begin{enumerate}
       \item Sistem je u funkciji.
       \item Klijent je prijavljen na sistem.
   \end{enumerate}\\
\hline  
    Postuslovi & \begin{enumerate}
        \item Klijent je odabrao željeni paket.
        \item Klijent se nalazi na stranici za on-line plaćanje.
    \end{enumerate}\\
\hline
    Osnovni tok & \begin{enumerate}
        \item Klijent pristupa delu sajta na kome se nalaze paketi koji su u ponudi (odnosno koji su trenutno aktuelni).
        \item Klijent klikom na paket dobija detaljne informacije o tom paketu.
        \item Klijent bira paket koji mu odgovara (po broju termina, ceni) klikom na dugme.
        % otvara se sledeća stranica pa onda...
        \item Klijent bira da li želi samostalni ili vođeni trening.
        \item Sistem 'odvodi' klijenta na stranicu za on-line plaćanje (izabrane usluge)
    \end{enumerate}\\
\hline
    Alternativni tokovi & \begin{itemize}
        \item[A4]  Klijent želi da zameni paket koji je odabrao: Klijent klikom na dugme za povratak vraća se korak nazad. Slučaj upotrebe se nastavlja u koraku 3.

    \end{itemize}\\
\hline
    Podtokovi & /\\
\hline
    Specijalni zahtevi & /\\
\hline
    Dodatne informacije & /\\
\hline
%\end{tabularx}
\caption{Odabir paketa za registrovanog korisnika (klijenta)} % needs to go inside longtable environment
%\label{tab:myfirstlongtable}
\end{longtable}
%\end{center}  
\end{document} 