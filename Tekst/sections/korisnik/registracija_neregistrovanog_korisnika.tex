\documentclass[../main.tex]{subfiles}

\begin{document}

\begin{center}
\caption{Registracija neregistrovanog korisnika}
\begin{tabularx}{\textwidth}{|1|X|}
\hline
    Kratak opis & Korisnik vrši registraciju tako što popunjava formular. Nakon toga sistem vrši validaciju podataka, registruje korisnika i šalje potvrdu korisniku.\\ \hline
    Učesnici & Neregistrovani korisnik koji želi da se registruje kako bi mogao da koristi usluge sistema. \\
\hline
   Preduslovi & Korisnik ima pristup internetu. Sistem je u funkciji. \\
   % Više nigde se ne pominje internet kao preduslov!
\hline  
    Postuslovi & Korisnik je uspešno registrovan i može da se prijavi na sistem i koristi njegove usluge. Baza podataka je ažurirana. \\
\hline
    Osnovni tok & \begin{enumerate}
    \item Korisnik otvara stranicu za registraciju.
	\item Korisnik unosi tražene podatke i kliknuo je na dugme za potvrdu/nastavak registracije.
	\item Sistem vrši validaciju podataka.
	\item Sistem čuva podatke i nalog obeležava kao privremen.
	\item Sistem šalje korisniku mejl sa privremenim (vremenski ograničenim) linkom za potvrdu i čeka na potvrdu.
	\item Korisnik proverava poštu i potvrđuje registraciju prateći link.
	\item Sistem obeležava nalog kao aktivan.
	\item Sistem obaveštava korisnika da je nalog uspešno sačuva (napravljen).\end{enumerate}\\
\hline
    Alternativni tokovi & \begin{itemize}
        \item [A3] Neuspešna validacija podataka: Sistem obaveštava korisnika da su podaci neispravni. Proces se nastavlja u koraku 2.
        \item [A5] Link za registraciju je istekao: Korisnik nije potvrdio registraciju. Sistem brise nalog. Proces se završava.
        \item [A6] E-mail nije stigao korisniku: Korisnik obaveštava sistem da mu ponovo pošalje e-mail. Slučaj se nastavlja u koraku 5.
    \end{itemize} \\
\hline
    Podtokovi & / \\
\hline
    Specijalni zahtevi & / \\
\hline
    Dodatne informacije & Potrebni podaci za registraciju su korisničko ime, lozinka, ime, prezime, e-mail adresa. \\
\hline
    
\end{tabularx}
\end{center}

\end{document}