\documentclass[../main.tex]{subfiles}

% Opis baze podataka

\begin{document}
U ovom poglavlju će biti prikazan opis modela baze podataka koji je predstaljen pomoću ER dijagrama. Svi entiteti koji se pojavljuju prate opise slučajeva upotrebe iz prethodnog poglavlja. Pojedini entiteti su specifični samo za pojedinačne slučajeve upotrebe, dok su entiteti kao što su \textit{Trening} i \textit{Klijent} zastupljeni u većini slučajeva. 

\subsection{Opisi entiteta}

Entitet \textit{Zaposleni} čuva osnovne podatke o zaposlenima i njegovu hijerahiju čine svi tipovi zaposlenog: \textit{Administrator}, \textit{Trener} i \textit{Recepcionar}. A \textit{Trener} predstavlja generalizaciju za entitete \textit{Personalni trener} i \textit{Grupni trener}

\textit{Lokacija} predstavlja lokaciju i osnovne podatke o konkretnoj teretani iz lanca, kao što su naziv, adresa, grad, radno vreme i broj telefona.

\textit{Trening} predstavlja informacije o jednom terminu treninga i od bitnih informacija čuva datum, vreme početka, vreme kraja i pomoću spoljašnjih ključeva informacije o lokaciji, vrsti treninga i treneru koji drži trening. \textit{Trening} se specijalizuje u entitete \textit{Personalni trening} i \textit{Grupni trening}

\\
\textit{Sala} predstavlja salu koja je odredjena lokacijom teretane i maksimalnim kapacitetom (maksimalan broj ljudi koji mogu trenirati u sali.

\textit{Grupni trening} je vrsta entita \textit{Trening} koji održava \textit{Trener} grupi \textit{Klijenata}. Opisuje ga id treninga, id trenera i maksimalni kapacitet grupnog treninga. Dok entitet \textit{Zakazani grupni}, kao i kod \textit{Treninga} ima informacije o vremenu, lokaciji i id grupnog treninga.
Entite \textit{Prijavljeni grupni} čuva informacije o kandidatima i njihovim prijavljenim grupni treninzima.
Entitet \textit{Probni trening} opisuje korisnike grupnih treninga koji prvi put prisustvuju i nisu registrovani korisnici. Zbog daljih provere, čuva se: JMBG, ime i prezime korisnika.

\textit{Personalni trening} čuva informaciju o tome koji klijent je prijavljen za taj trening. Ukoliko nijedan klijent nije prijavljen, polje koje odgovara klijentu ću biti null. A informacije o treneru i terminu održavanja se mogu dobiti iz entiteta \textit{Trening} sa kojim je povezan pomoću identifikatora treninga. Entitet \textit{Raspored personalnog trenera} za svakog personalnog trenera čuva informacije o zakazanim treninzima.

Takođe, svaki klijent ima svoj raspored treninga. Informacije o tome se čuvaju u entitetu \textit{Raspored klijenta} koji sadrži id datog klijenta i identifikatore treninga (i personalnih i grupnih) na koje se on prijavio.

Pomoću agregiranog entiteta \textit{Dolasci} vodi se evidencija o dolasku klijenata na treninge.

\\
Za grupu slučajeva upotrebe za takmičenja centralni entitet je \textit{Takmičenje}. On kao atribute ima jedinstveni identifikator koji čini primarni ključ, informacije o datumu, vremenu početka, vremenu kraja, dodatnim informacijama i nagradama i lokaciji. Informacije o disciplinama se dobijaju pomoću veze sa entitetom \textit{Discipline} koji čuva sve moguće discipline. S obzirom da na jednom takmičenju može biti više sudija, a sudije čine treneri teretane, kreiran je poseban entitet \textit{Sudije} koji čuva za svako takmičenje ko je sudija, a detaljnije informacije o osobama se mogu dobiti na osnovu veze sa entitetom \textit{Trener}. \textit{Takmičar} je poseban entitet koji se kreira pri registraciji novog takmičara i sadrži osnovne informacije koje se uzimaju popunjavanjem formulara.Njega jedinstveno određuje kombinacija broja prijave i id takmičenja. Takođe, povezan je vezom više-više sa \textit{Takmičenje\_discipline} koji predstavlja skup disciplina na koje je moguće prijaviti se za neko takmičenje. U slučaju da se takmičenje otkaže kreira se entitet \textit{Participacija} koji za svakog takmičara sa datog takmičenja čuva informacije na koji način mu vratiti novac i informaciju da li je vraćen - koja je inicijalno ne. 

Što se tiče obuke za licencu, centralni entitet te grupe slučajeva je \textit{Program obuke za licencu}. On sadrži informacije o tipu licence koja se dobija, lokaciji na kojoj se program održava, časovima, trenerima koji drže obuku i završnom ispitu. Veza između \textit{Programa obuke za licencu} i \textit{Trenera} je više-više, pošto jednu obuku može da drži više trenera, a jedan trener može da drži više različitih obuka. \textit{Ispit} sadrži informacije o datumu, vremenu početka i kraja završnog ispita, klijentu koji polaže ispit, njegovoj oceni, kao i o komisiji koja je dala tu ocenu. Komisiju čini više trenera, a jedan trener može da bude član više komisija, pa je zbog toga između entiteta \textit{Komisija} i \textit{Trener} uspostavljena veza više-više. Evidencija o prijavama za obuku vodi se u agregiranom entitetu \textit{Prijava za program obuke za licencu} koji sadrži i dokumenta koja klijent mora da preda kako bi mogao da pohađa obuku.


\end{document}
