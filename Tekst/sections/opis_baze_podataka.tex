\documentclass[../main.tex]{subfiles}

% Opis baze podataka

\begin{document}
U ovom poglavlju će biti prikazan opis modela baze podataka koji je predstaljen pomoću ER dijagrama. Svi entiteti koji se pojavljuju prate opise slučajeva upotrebe iz prethodnog poglavlja. Pojedini entiteti su specifični samo za pojedinačne slučajeve upotrebe, dok su entiteti kao što su \textit{Trening} i \textit{Klijent} zastupljeni u većini slučajeva. 

\subsection{Opisi entiteta}

Entitet \textit{Zaposleni} čuva osnovne podatke od zaposlenima i njegovu hijerahiju čine svi tipovi zaposlenog: \textit{Administrator}, \textit{Trener} i \textit{Recepcionar}. 

\textit{Trening} predstavlja informacije o jednom terminu treninga i od bitnih informacija čuva datum, vreme početka, vreme kraja i pomoću spoljašnjih ključeva informacije o lokaciji, vrsti treninga i treneru koji drži trening. 

\textit{Lokacija} predstavlja lokaciju i osnovne podatke o konkretnoj teretani iz lanca, kao što su naziv, adresa, grad, radno vreme i broj telefona. 



Za grupu slučajeva upotrebe za takmičenja centralni entitet je \textit{Takmičenje}. On kao atribute ima jedinstveni identifikator koji čini primarni ključ, informacije o datumu, vremenu početka, vremenu kraja, dodatnim informacijama i nagradama i lokaciji. Informacije o disciplinama se dobijaju pomoću veze sa entitetom \textit{Discipline} koji čuva sve moguće discipline. S obzirom da na jednom takmičenju može biti više sudija, a sudije čine treneri teretane, kreiran je poseban entitet \textit{Sudije} koji čuva za svako takmičenje ko je sudija, a detaljnije informacije o osobama se mogu dobiti na osnovu veze sa entitetom \textit{Trener}. \textit{Takmičar} je poseban entitet koji se kreira pri registraciji novog takmičara i sadrži osnovne informacije koje se uzimaju popunjavanjem formulara.Njega jedinstveno određuje kombinacija broja prijave i id takmičenja. Takođe, povezan je vezom više-više sa \textit{Takmičenje\_discipline} koji predstavlja skup disciplina na koje je moguće prijaviti se za neko takmičenje. U slučaju da se takmičenje otkaže kreira se entitet \textit{Participacija} koji za svakog takmičara sa datog takmičenja čuva informacije na koji način mu vratiti novac i informaciju da li je vraćen - koja je inicijalno ne. 



\end{document}
