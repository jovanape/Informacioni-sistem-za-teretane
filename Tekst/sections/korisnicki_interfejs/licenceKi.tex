Delu aplikacije koji je vezan za dobijanje licenci mogu da pristupe administrator, klijent i recepcioner. Svako od njih može da vidi opise postojećih programa obuke. A dodatno, u zavisnosti od toga koji korisnik sistema je ulogovan, ova sekcija nudi različite opcije.

Administrator može da vidi dugme ``Dodaj novi program obuke`` pomoću koga se otvara formular za unos informacija o novom programu obuke. Nakon popunjavanja formulara može odabrati opciju ``Dodaj`` čime se proverava da li su uneti svi potrebni podaci. Ukoliko nisu, administrator se obaveštava o tome i daje mu se mogućnost da popuni preostala polja. Ukoliko u bilo kom trenutku, administrator želi da odustane od dodavanja novog programa, može kliknuti dugme "Odustani" čime se brišu sve prethodno unete informacije i zatvara se formular. Izgled ovog segmenta, može se videti na slikama \ref{fig:novi_program1} i \ref{fig:novi_program2}.
Administrator ima i opciju unosa rezultata ispita. Klikom na dugme ``Objavi rezultate ispita`` ispod opisa nekog programa obuke, prelazi se na stranicu koja sadrži formular za unos ocena. Na stranici će već biti prikazan spisak klijenata koji su prijavljeni na taj program, a administrator samo treba u odgovarajuća polja da unese njihove ocene i informaciju o tome da li dobijaju licencu. Izgled formulara je prikazan na slici \ref{fig:admin_ispiti}.

\begin{figure}[!ht]
\begin{center}
\includegraphics[scale=0.30]{sections/korisnicki_interfejs/screenshots/licenca-admin.png}
\end{center}
\caption{Dodavanje novog programa obuke}
\label{fig:novi_program1}
\end{figure}

\begin{figure}[!ht]
\begin{center}
\includegraphics[scale=0.60]{sections/korisnicki_interfejs/screenshots/licenca-admin-nov-program.png}
\end{center}
\caption{Dodavanje novog programa obuke - forma}
\label{fig:novi_program2}
\end{figure}

\begin{figure}[!ht]
\begin{center}
\includegraphics[scale=0.45]{sections/korisnicki_interfejs/screenshots/licenca-admin-ispiti.png}
\end{center}
\caption{Formular za unos rezultata ispita}
\label{fig:admin_ispiti}
\end{figure}

Klijent može da se prijavi na program obuke klikom na dugme ``Prijavi se``, ispod koga će se otvoriti formular koji je potrebno popuniti. Klijent u svakom trenutku može da odustane od popunjavanja prijave klikom na dugme "Odustani". Klijent takođe može da pogleda rezultate ispita svakog od programa klikom na dugme ``Rezultati ispita``, čime se prelazi na stranicu koja sadrži sve potrebne informacije za rezultate datog programa. Formular za prijavu se nalazi na slici \ref{fig:licenca_klijent_prijava}, a stranica za prikaz rezultata na slici \ref{fig:licenca_klijent_ispiti}.

\begin{figure}[!ht]
\begin{center}
\includegraphics[scale=0.45]{sections/korisnicki_interfejs/screenshots/licenca-klijent-prijava.png}
\end{center}
\caption{Formular za prijava klijenta na program obuke}
\label{fig:licenca_klijent_prijava}
\end{figure}

\begin{figure}[!ht]
\begin{center}
\includegraphics[scale=0.25]{sections/korisnicki_interfejs/screenshots/licenca-klijent-ispiti.png}
\end{center}
\caption{Rezultati ispita}
\label{fig:licenca_klijent_ispiti}
\end{figure}


Recepcioner ima mogućnost prijave klijenta na program obuke. Klikom na dugme ``Prijavi klijenta`` otvara se odgovarajući formular, koji izgleda slično kao i formular za samostalnu prijavu klijenta.