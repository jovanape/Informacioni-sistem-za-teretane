\documentclass[../main.tex]{subfiles}

\begin{document}

Kao što je navedeno u prethodnom poglavlju ovaj informacioni sistem je osmišljen kao veb aplikacija i samim tim će u nastavku biti prikazan izgled korisničkog interfejsa kreiranog u Angular-u.
Vođeno je računa da aplikacija bude pristupačna korisnicima i laka za upotrebu. Radi što boljeg korisničkog iskustva navigacija je kreirana na samom vrhu, postoji dugme koje u bilo kom trenutku korisnika može da vrati na vrh strane, u zavisnosti od tipa korisnika koji je ulogovan drugačiji je prikaz stranica u navigaciji. Takođe, formulari su laki za upotrebu i zahtevaju samo neophodne podatke od korisnika, dok većina polja zahteva ne zahteva unošenje teksta već samo odabir.

%Ovde da bude login, registracija i početna

\subsection{Početna strana}
\subfile{pocetnaKi}


\subsection{Paketi}
\subfile{paketiKi}


\subsection{Igraonica}
\subfile{igraonicaKi}


\subsection{Kupovina paketa}
U ovoj sekciji biće opisan deo korisničkog interfejsa koji služi za kupovinu paketa za teretanu (podsekcija \ref{subsec:kp_teretane}) i igraonicu (podsekcija \ref{subsec:kp_igraonice}). Forma preko koje se vrši plaćanje odabranih paketa biće opisana u podsekciji \ref{subsec:placanje}.

\subsubsection{Kupovina paketa teretane}
\label{subsec:kp_teretane}
\subfile{forma_teretana}

\subsubsection{Kupovina paketa igraonice}
\label{subsec:kp_igraonice}
\subfile{forma_igraonica}

\subsubsection{Plaćanje paketa}
\label{subsec:placanje}
\subfile{forma_placanje}


\subsection{Administracija}
\subfile{administracijaKi}

\subsection{Treninzi}
\subfile{treninziKi}

\subsection{Licence}
\subfile{licenceKi}

\subsection{Evidentiranja}
\subfile{evidentiranjaKi}


\subsection{Takmičenja}
\subfile{takmicenjaKi}

Celokupan prikaz svih stranica može se pronaći na adresi: \href{https://github.com/jovanape/Informacioni-sistem-za-teretane/tree/main/screenshot-ovi_web_aplikacije}{Slike korisničkog interfejsa}

\end{document}